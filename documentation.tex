\documentclass[a4paper,12pt]{article}
\usepackage[utf8]{inputenc}
\usepackage{graphicx}
\usepackage{amsmath}
\usepackage{hyperref}
\usepackage{enumitem}
\usepackage{biblatex}

\addbibresource{documentation.bib}


\title{An Evaluation of the current State of web server development in Rust and Java}
\author{Marc Matija}
\date{\today}
 
\begin{document}
	\maketitle
	\vspace{2cm}
	\begin{abstract}
		This paper aims to evaluate of the current state of web server development in Rust and Java. 
		In doing so, it will cover the basics of web server development in both languages, the tools and libraries 
		available, and the advantages and disadvantages of using each language for web server development. 
		Furthermore it will include a comparison of the two languages in terms of performance, ease of 
		use, and other factors that are important for web server development.
	\end{abstract}
	
	\newpage
	\tableofcontents
	\newpage
	
	\section{Introduction}
	\label{sec:introduction}
	Web server development is an important aspect of modern software development as they are 
	used to host websites, web applications, and other online services.
	One of the most popular languages for this is Java and the Spring framework.
	Java is a mature and widely used programming language that is known for its performance, scalability, and reliability.
	However, in recent years, Rust has emerged as a promising alternative to Java for web server development as seen at Discord 
	\cite{Discord}, Cloudflare\cite{Cloudflare_Pingora}, and other companies. 
	Rust is in stark contrast to Java, is a systems programming language that is designed for performance, safety, and concurrency.
	This paper aims to evaluate the current state of web server development in Rust and Java, and to provide an overview of 
	the tools, libraries, and best practices for web server development in both languages.
	
	\subsection{An Overview of Spring Boot}
	\label{subsec:spring_boot}
	Spring Boot is a popular Java framework for building web applications and microservices used by companies like 
	Deutsche Bahn\cite{DB_Job_Description}, Udemy \cite{Techstack_Udemy} and many others\cite{Spring_Boot_stackshare}.
	Built on top of the Spring framework, Spring Boot provides a set of tools and libraries that make it easy to build
	web applications and microservices in Java. It finds wide adoption in the industry due to its ease of use, maintanability 
	and scalability, aswell as the vast amount of libraries and tools available for it.

	\subsection{An Overview of Actix Web}
	\label{subsec:actix_web}
	Actix is a Rust framework for building backend services in rust and the one currently one of the most actively developed framworks,
	within the Rust ecosystem. It provides provides a very similar feature set to Spring Boot making it a suitable candidate for comparison.


	\section{Specifications for the web server}
	\label{sec:example_project}
	In order to evaluate the current state of web server development in Rust and Java, I will build a simple web server in both languages
	using the previously mentioned frameworks. The web server will feature a REST api that allows users to create channels, add messages to channels,
	and retrieve messages from said channels. In addition, the web server will feature a websocket endpoint that allows users to subscribe to channels
	and receive messages in real-time.

	\subsection{REST API}
	\subsection{Websocket API}
	\subsection{Database Schema}
	\subsection{Authentication}

	\section{Implementation in Java}
	\label{sec:java_implementation}

	\section{Implementation in Rust}
	\label{sec:rust_implementation}

	\section{Comparison}
	\label{sec:comparison}
	\subsection{Developer Experience}
	\subsection{Performance}
	
	\section{Conclusion}

	\section{Sources}
	\label{sec:Sources}
	\begin{itemize}
		\item \href{https://rust-edu.org/resources/}{Rust Edu}
		\item \href{https://doc.rust-lang.org/book/}{The Rust Programming Language}
		\item \href{https://doc.rust-lang.org/std/}{Rust Standard Library}
		\item \href{https://doc.rust-lang.org/cargo/}{Cargo Guide}
		\item \href{https://doc.rust-lang.org/rust-by-example/}{Rust by Example}
		\item \href{https://doc.rust-lang.org/rustdoc/}{Rustdoc}
		\item \href{https://actix.rs/}{Actix}
	\end{itemize}
	\printbibliography[title={Whole bibliography}]
\end{document}